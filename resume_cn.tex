%!TEX TS-program = xelatex

\documentclass[9pt]{article}
\usepackage{xeCJK}
\newfontfamily\bodyfont[]{Helvetica Neue}
\newfontfamily\thinfont[]{Helvetica Neue UltraLight}
\newfontfamily\headingfont[]{Helvetica Neue Condensed Bold}

\defaultfontfeatures{Mapping=tex-text}
\setmainfont[Mapping=tex-text, Color=textcolor]{Helvetica Neue Light}
\setCJKmainfont[BoldFont={SimHei}, ItalicFont={Adobe Kaiti Std}]{Adobe Kaiti Std}
\usepackage{fullpage}
\usepackage{amsmath}
\usepackage{amssymb}
\usepackage[usenames]{color}

\leftmargin=0.25in
\oddsidemargin=0.25in
\textwidth=6.0in
\topmargin=-0.25in
\textheight=9.25in

\raggedright

\pagenumbering{arabic}

\def\bull{\vrule height 0.8ex width .7ex depth -.1ex }
% DEFINITIONS FOR RESUME

\newenvironment{changemargin}[2]{%
  \begin{list}{}{%
    \setlength{\topsep}{0pt}%
    \setlength{\leftmargin}{#1}%
    \setlength{\rightmargin}{#2}%
    \setlength{\listparindent}{\parindent}%
    \setlength{\itemindent}{\parindent}%
    \setlength{\parsep}{\parskip}%
  }%
  \item[]}{\end{list}
}

\newcommand{\lineover}{
	\begin{changemargin}{-0.05in}{-0.05in}
		\vspace*{-8pt}
		\hrulefill \\
		\vspace*{-2pt}
	\end{changemargin}
}

\newcommand{\header}[1]{
	\begin{changemargin}{-0.5in}{-0.5in}
		\scshape{#1}\\
  	\lineover
	\end{changemargin}
}

\newcommand{\contact}[4]{
	\begin{changemargin}{-0.5in}{-0.5in}
		\begin{center}
			{\Large \scshape {#1}}\\ \smallskip
			{#2}\\ \smallskip
			{#3}\\ \smallskip
			{#4}\smallskip
		\end{center}
	\end{changemargin}
}

\newenvironment{body} {
	\vspace*{-16pt}
	\begin{changemargin}{-0.25in}{-0.5in}
  }
	{\end{changemargin}
}

\newcommand{\school}[4]{
	\textbf{#1} \hfill \emph{#2\\}
	#3\\
	#4\\
}

% END RESUME DEFINITIONS

\begin{document}

%%%%%%%%%%%%%%%%%%%%%%%%%%%%%%%%%%%%%%%%%%%%%%%%%%%%%%%%%%%%%%%%%%%%%%%%%%%%%%%%
% Name
\contact{肖智博}{xiaozhibo@gmail.com}{186 426 22311}


%%%%%%%%%%%%%%%%%%%%%%%%%%%%%%%%%%%%%%%%%%%%%%%%%%%%%%%%%%%%%%%%%%%%%%%%%%%%%%%%
% Objective
% \header{Objective}

% \begin{body}
% 	\vspace{14pt}
% 	To get a job
% \end{body}

% \smallskip
% \medskip


%%%%%%%%%%%%%%%%%%%%%%%%%%%%%%%%%%%%%%%%%%%%%%%%%%%%%%%%%%%%%%%%%%%%%%%%%%%%%%
\header{\Large{教育经历}}

\begin{body}
   \vspace{14pt}
	\textbf{博士}{} \hfill \emph{2009.9-2014.6}{} \\
	\emph{排序主题模型}, 信息科学技术学院,大连海事大学{} \\
  \medskip
	% \vspace{14pt}
	\textbf{硕士}{} \hfill \emph{2007.9-2009.7}{} \\
	\emph{面向科学基金管理的数据仓库和展现工具研究}, 信息科学技术学院,大连海事大学{} \\
  \medskip
	\textbf{本科} \hfill \emph{2003.9-2007.7} \\
	数学系,大连海事大学\\
\end{body}

\smallskip
\medskip
%%%%%%%%%%%%%%%%%%%%%%%%%%%%%%%%%%%%%%%%%%%%%%%%%%%%%%%%%%%%%%%%%%%%%%%%%%%%%%
\header{\Large{发表的期刊论文}}
\begin{body}
	\vspace{14pt}
	1. \textbf{Zhi-Bo Xiao}, Feng Che, Enuo Miao, and Mingyu Lu. “Increasing Serendipity of Recommender System with Ranking Topic Model.” Applied Mathematics \& Information Sciences. 2014, to appear. \textbf{SCI}\\
	\smallskip
	2. \textbf{Zhi-Bo Xiao}, Feng Che, Enuo Miao, and Mingyu Lu. “CorrRank: Correlation Based Ranking Topic Model.” Journal of Computational Information Systems. 2014,5. \textbf{EI}\\
	\smallskip
	3. \textbf{Zhi-Bo Xiao}, Ping Li, Yingliang Fu, and Mingyu Lu. “Does More Randomness Help Increasing Diversity in Ensemble Learning?.” Journal of Convergence Information Technology 7 (19): 309–318. \textbf{EI}\\
	\smallskip
	4. \textbf{肖智博},吴镝,车丰,鲁明羽. CorrSum:基于排序主题模型的多文档自动文摘算法. 模式识别与人工智能. 已录用\\
	\smallskip
	5. Jun, Wu, \textbf{Zhi-Bo Xiao}, Wang Huai-Shuai, and Hong Shen. “Learning with Both Unlabeled Data and Query Logs for Image Search.” Computers \& Electrical Engineering. \textbf{SCI}\\
	\smallskip
	6. Lin, Zheng-Kui, Jun Wu, \textbf{Zhi-Bo Xiao}, Jing Duan, and Ming-Yu Lu. “Unified Learning Paradigm for Image Retrieval.” International Journal of Innovative Computing Information and Control 7 (8): 4977–4987. \textbf{SCI}\\
	\smallskip
	7. 刘娜,\textbf{肖智博},鲁明羽. 基于形态学的单词-文档谱聚类方法. 南京大学学报(自然科学版). 2012 (02)\\
	\smallskip
	8. 刘娜,\textbf{肖智博},鲁明羽. 基于模糊K-调和均值的单词-文档谱聚类方法. 控制与决策. 2012 (04)\\
	\smallskip
	9. 刘权,\textbf{肖智博},鲁明羽.   面向科学基金管理数据数据仓库概念模型设计. 计算机工程与应用. 2009 (36)

\end{body}

\smallskip
\medskip
%%%%%%%%%%%%%%%%%%%%%%%%%%%%%%%%%%%%%%%%%%%%%%%%%%%%%%%%%%%%%%%%%%%%%%%%%%%%%%
\header{\Large{参与过的基金项目}}

\begin{body}
	\vspace{14pt}
	大数据环境下稀疏主题模型理论及其应用研究(国家自然科学基金面上项目,61370070)\\ 资助金额:75万,资助时间:2014年至2017年。\\
	\textbf{独立撰写基金申请书,项目管理,算法研制}\\
	\medskip
	排序主题模型及其应用研究(国家自然科学基金面上项目,61272369)\\	资助金额:80万,资助时间:20113年至2016年。\\
	\textbf{独立撰写基金申请书,项目管理,算法研制}\\
	\medskip
	基于半监督学习和集成学习的文本分类方法研究(国家自然科学基金面上项目,61073133)\\ 资助金额:32万,资助时间:2011年至2013年。\\
	\textbf{参与撰写申请书,主要技术负责人,项目统筹}\\
	\medskip
	GIS方法在科学基金统计分析中的应用研究(国家自然科学基金专项项目, J0724003)\\
	资助金额:8万,资助时间:2004年。\\
	\textbf{主要技术负责人}\\
	\medskip
	基于图像挖掘的冠心病循证医疗辅助诊断系统(大连市科技计划项目)\\
	\textbf{参与项目申请}
\end{body}

\smallskip
\medskip
\newpage

%%%%%%%%%%%%%%%%%%%%%%%%%%%%%%%%%%%%%%%%%%%%%%%%%%%%%%%%%%%%%%%%%%%%%%%%%%%%%%
\header{\Large{参与的横向课题及其他工作}}

\begin{body}
	\vspace{14pt}
	\textbf{审稿人},《计算机科学》 \hfill{} \emph{2010年至今}\\
	\smallskip
	\textbf{审稿人},《计算机工程与应用》 \hfill{} \emph{2010年至今}\\
	\smallskip
	\textbf{主要项目负责人},海尔统帅电视销售决策支持技术项目推动 \hfill{} \emph{2010.12 - 2011.03} \\
	\smallskip
	\textbf{主要技术负责人}, 上海通用汽车公司客户行为分析数据挖掘项目\hfill {} \emph{2010年 11月 - 2010年 12月}\\
	\smallskip
	\textbf{主要技术负责人,联络人},知识管理在虚拟装配中的应用,大连海事大学机电学院合作项目 \\\hfill{}  \emph{2010年 9月 - 2011年 9月}\\
	\smallskip
	\textbf{参与研究},大连海事大学系统生物学研究所合作项目 \hfill{} \emph{2009年 - 2012年}\\
	\smallskip
    \textbf{培训讲师}	,赛科公司智能巡检系统 \hfill{} \emph{2008年 4 - 5月}\\
    \smallskip
    \textbf{主要技术负责人},天津港辅助决策系统 \hfill{} \emph{2007年 9 - 12月} \\
    \smallskip
    \textbf{主要技术负责人},大连软件外包产业研究 \hfill{} \emph{2007年 7月} \\
\end{body}

\smallskip
\medskip



%%%%%%%%%%%%%%%%%%%%%%%%%%%%%%%%%%%%%%%%%%%%%%%%%%%%%%%%%%%%%%%%%%%%%%%%%%%%%%
\header{\Large{独立指导的研究生}}

\begin{body}
	\vspace{14pt}
	\textbf{车丰},基于排序主题模型和众包策略的学术论文推荐系统研究 \hfill{} \emph{2014年毕业}\\
	\smallskip
	\textbf{赵云},主题模型的评价方法研究 \hfill{} \emph{2014年毕业}\\
	\smallskip
	\textbf{乔桢},基于半监督学习的集成学习多样性研究 \hfill{} \emph{2014年毕业}\\
	\smallskip
	\textbf{徐宇婷},基于排序主题模型和众包策略的多文档自动文摘方法研究 \hfill{} \emph{2014年毕业}\\
	\smallskip
	\textbf{李庆丰},基于主题模型的多文档自动文摘方法研究 \hfill{} \emph{2013年毕业}\\
	\smallskip
	\textbf{黄泽明},基于主题模型的学术论文推荐系统研究 \hfill{} \emph{2013年毕业}\\
	\smallskip
	\textbf{刘文强},语音识别技术在智能家居中的研究与应用 \hfill{} \emph{2013年毕业}\\
	\smallskip
	\textbf{李平},集成学习中差异性控制方法研究 \hfill{} \emph{2012年毕业}\\
	\smallskip
	\textbf{周帅},基于数据挖掘技术的犯罪相关因素分析 \hfill{} \emph{2012年毕业}\\
	\smallskip
	\textbf{钱新宇},基于实例推理的虚拟装配序列规划研究 \hfill{} \emph{2011年毕业}\\
	\smallskip
	\textbf{王询},面向航运领域的文本分类系统 \hfill{} \emph{2011年毕业}\\
	\smallskip
	\textbf{洪锦东},基于程序挖掘的构件组装建模方法及工具研究 \hfill{} \emph{2010年毕业}\\
	\smallskip
\end{body}

\end{document}
