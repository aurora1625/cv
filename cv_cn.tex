%!TEX TS-program = xelatex
\documentclass[]{friggeri-cv}
\addbibresource{bibliography.bib}

\begin{document}
\header{肖}{ 智博}
       {机器学习博士}


% in the aside, each new line forces a line break
\begin{aside}
  \section{关于}
    B-106
    信息科学技术学院
    大连, 辽宁 116026
    中国
    ~
    \href{mailto:xiaozhibo@gmail.com}{xiaozhibo@gmail.com}
    %\href{http://friggeri.net}{http://friggeri.net}
    %\href{http://facebook.com/adrien}{fb://adrien}
  \section{语言水平}
    iBT TOEFL: 99
    IELTS: 7.5
    TOEIC: 905
    CET-6: 602
    CET-4: 632
  \section{编程语言}
    {\color{red} $\varheartsuit$} Python
    R, \LaTeX, C
\end{aside}

\section{研究兴趣}

教学, 机器学习,数据挖掘,推荐系统,信息检索,自动文摘,自然语言处理


\section{教育背景}


\begin{entrylist}
  \entry
    {2009至今}
    {博士研究生}
    {大连海事大学}
    {论文题目:\emph{排序主题模型}}
  \entry
    {2007–2009}
    {工学硕士}
    {大连海事大学}
    {计算机学院\\
    数据挖掘专业}
  \entry
    {2007–2009}
    {理学学士}
    {大连海事大学}
    {数学系}
\end{entrylist}


\section{发表期刊}


\printbibsection{article}{期刊发表文章}
\begin{refsection}
  \nocite{*}
  \printbibliography[sorting=chronological, type=inproceedings, title={international peer-reviewed conferences/proceedings}, notkeyword={france}, heading=subbibliography]
\end{refsection}
\begin{refsection}
  \nocite{*}
  \printbibliography[sorting=chronological, type=inproceedings, title={local peer-reviewed conferences/proceedings}, keyword={france}, heading=subbibliography]
\end{refsection}
\printbibsection{misc}{other publications}
\printbibsection{report}{research reports}

\section{基金项目}

\begin{entrylist}
  \entry
    {2014至今}
    {国家自然科学基金面上项目, 61370070, 75万}
    {主要技术负责人,项目统筹}
    {\emph{大数据环境下稀疏主题模型理论及其应用研究}}
  \entry
    {12.11-13.03}
    {国家自然科学基金面上项目, 61370070, 75万}
    {独立撰写申请书}
    {\emph{大数据环境下稀疏主题模型理论及其应用研究}}
  \entry
    {2013至今}
    {国家自然科学基金面上项目, 61272369, 80万}
    {主要技术负责人,项目统筹}
    {\emph{排序主题模型及其应用研究}}
  \entry
    {11.11-12.03}
    {国家自然科学基金面上项目, 61272369, 80万}
    {独立撰写申请书}
    {\emph{排序主题模型及其应用研究}}
  \entry
    {2011-2013}
    {国家自然科学基金面上项目, 61073133, 32万}
    {主要技术负责人,项目统筹}
    {\emph{基于半监督学习和集成学习的文本分类方法研究}}
  \entry
    {2010.6}
    {大连市科技计划项目}
    {参与项目申请}
    {基于图像挖掘的冠心病循证医疗辅助诊断系统}
  \entry
    {2010.3}
    {国家自然科学基金面上项目, 61073133, 32万}
    {参与撰写}
    {\emph{基于半监督学习和集成学习的文本分类方法研究}}
  \entry
    {2008-2010}
    {国家自然科学基金面上项目, 60773084, 32万}
    {主研人员}
    {\emph{分析挖掘冠心病中医诊疗临床规律的智能技术研究}}
  \entry
    {2008-2010}
    {高等学校博士学科点专项科研基金, 20070151009}
    {主研人员}
    {\emph{循证医学文献的自动分类方法研究}}
  \entry
    {2004}
    {国家自然科学基金专项项目, J0724003, 8万}
    {主要技术负责人}
    {\emph{GIS方法在科学基金统计分析中的应用研究}}
\end{entrylist}

\section{其他项目以及工作}
\begin{entrylist}
  \entry
    {2010至今}
    {审稿}
    {核心期刊}
    {计算机工程与应用}
  \entry
    {2009-2010}
    {系统生物学研究所合作项目}
    {参与研究,合作联系}
    {}
  \entry
    {2010-2011}
    {合作项目,机电学院}
    {前期调研,需求分析,系统设计,算法设计}
    {知识管理在虚拟装配中的应用}
  \entry
    {10.12-11.03}
    {海尔统帅电视销售决策支持技术项目推动}
    {主要项目负责人}
    {海尔电视,品牌推广}
  \entry
    {10.11-12}
    {SGM CEM 数据挖掘项目}
    {主要技术负责人}
    {全球500强汽车企业,用户行为研究,再购预测}
  \entry
    {2008}
    {赛科公司智能巡检系统}
    {培训讲师}
    {}
  \entry
    {2007}
    {天津港辅助决策系统}
    {主要技术负责人}
    {}
  \entry
    {2007}
    {大连软件外包产业研究}
    {主要技术负责人}
    {}
\end{entrylist}

\section{指导研究生、本科生毕业设计}
\begin{entrylist}
  \entry
    {2013}
    {李庆丰}
    {导师:鲁明羽}
    {基于主题模型的多文档自动文摘方法研究}
  \entry
    {2013}
    {黄泽明}
    {导师:鲁明羽}
    {基于主题模型的学术论文推荐系统研究}
  \entry
    {2013}
    {刘文强}
    {导师:付英亮}
    {语音识别技术在智能家居中的研究与应用}
  \entry
    {2012}
    {李平}
    {导师:付英亮}
    {集成学习中差异性控制方法研究}
  \entry
    {2012}
    {周帅}
    {导师:鲁明羽}
    {基于数据挖掘技术的犯罪相关因素分析}
  \entry
    {2011}
    {钱新宇}
    {导师:鲁明羽}
    {基于实例推理的虚拟装配序列规划研究}
  \entry
    {2011}
    {王询}
    {导师:鲁明羽}
    {面向航运领域的文本分类系统}
  \entry
    {2010}
    {洪锦东}
    {导师:鲁明羽}
    {基于程序挖掘的构件组装建模方法及工具研究}
  \entry
    {2010}
    {徐东坤}
    {导师:鲁明羽}
    {面向互联网的构件获取技术研究}

\end{entrylist}


\end{document}
